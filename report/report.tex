\documentclass[conference]{IEEEtran}
\usepackage[utf8]{inputenc}
\usepackage{graphicx}
\usepackage{amsmath}
\usepackage{cite}
\usepackage{url}

\title{ROSBot Pro 2.0 Swarm Behaviours}
\author{
\IEEEauthorblockN{Haocheng Wang\IEEEauthorrefmark{1}, Huy Do\IEEEauthorrefmark{2}, Shengqing Jin\IEEEauthorrefmark{3}, Yenhsu Chou\IEEEauthorrefmark{4}}
\IEEEauthorblockA{COSC2781/COSC2814 Programming Autonomous Robots\\
RMIT University, City Campus\\
\IEEEauthorrefmark{1}Student ID: 3789513,
\IEEEauthorrefmark{2}Student ID: 3894502,
\IEEEauthorrefmark{3}Student ID: 3687137,
\IEEEauthorrefmark{4}Student ID: 3961072}
}

\begin{document}

\maketitle

\begin{abstract}
This report presents the development, implementation, and evaluation of swarm behaviours in ROSBot Pro 2.0 robots.
\end{abstract}

\section{Introduction}
\label{sec:intro}
Swarm behaviour in autonomous robotics involves the coordination of multiple robots to perform tasks collectively. This project focuses on implementing and evaluating swarm behaviours in ROSBot Pro 2.0 robots.

\section{Related Work}
\label{sec:related}
The study of swarm robotics includes various algorithms and communication protocols. We chose to use TCP (Transmission Control Protocol) as the communication method between robots.

\section{Methodology}
\label{sec:method}
The methodology for this project involves the setup of simulation environments, the implementation of algorithms, and the testing of these algorithms both in simulation and real-world environments.

\subsection{Communication Protocol: Using TCP}
Using TCP (Transmission Control Protocol) was chosen for this project due to several advantages:
\begin{itemize}
    \item \textbf{Reliability}: TCP ensures that data sent from one robot reaches the destination accurately and in the same order it was sent, which is crucial for coordinated actions in swarm behaviour.
    \item \textbf{Connection-Oriented}: TCP establishes a continuous and dedicated connection, maintaining a steady flow of information necessary for real-time decisions.
    \item \textbf{Error Checking and Correction}: TCP handles error checking and correction automatically, ensuring complete and accurate information.
    \item \textbf{Congestion Control}: TCP effectively manages network congestion, maintaining efficient communication within the swarm.
\end{itemize}

\subsection{Leader-Follower Interaction}
Two interaction modes were considered:
\begin{itemize}
    \item \textbf{Direct Copy}: This mode involves one robot copying the movements of another directly.
    \item \textbf{Position Tracking and Following}: This mode involves a follower robot receiving positional data from a leader robot and adjusting its movements accordingly.
\end{itemize}

The implementation of the position tracking and following mode involved the following steps:
\begin{enumerate}
    \item Establishing a TCP connection between the leader and follower robots.
    \item Receiving the leader's positional data and updating the follower's target position.
    \item Calculating the angle and direction to the target position and adjusting the follower's movement accordingly.
    \item Continuously monitoring and adjusting the follower's position to maintain the desired distance from the leader.
\end{enumerate}

\section{Results}
\label{sec:results}
result

\section{Evaluation and Discussion}
\label{sec:evaluation}
evaluation

\section{Conclusion}
\label{sec:conclusion}
conclusion

\section*{References}
\bibliographystyle{IEEEtran}
\bibliography{references}

\end{document}
